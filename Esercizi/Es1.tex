\begin{alphaparts}
    %-------------------------------------A---------------------------------------
   \questionpart
   Per la dinamica \(P\) (non lazy) distinguiamo 2 casi:
   \begin{description}
       \item[\(\bullet\)] Se \(\frac{n}{2}\) è dispari allora il grafo \textbf{non} è aperiodico, pertanto la convergenza della dinamica \(P\) non è garantita.
       \item[\(\bullet\)] Se \(\frac{n}{2}\) è pari allora siano 
       \[p_1 = \{0,1,2, \dots ,n - 1, 0\}\] 
       \[p_2 = \{0,1,2, \dots \frac{n}{2}, 0\}\]
       due cicli di lunghezza rispettivamente \(n\) e \(\frac{n}{2} + 1\).
       \[\frac{n}{2}  \quad \text{pari} \quad \implies\, n= 4k,\,k\in \mathbb{N}\]
       Abbiamo quindi trovato due cicli di lunghezza \(4k\) e \(2k + 1\) che sono numeri coprimi tra loro, pertanto il grafo è aperiodico e la convergenza è garantita. In particolare
       \[x(t) \xrightarrow[]{t \to \infty} (\pi 'x(0)) \mathbf{1}\]
   \end{description} 
   Poiché il grafo \(R_n\) è regolare, allora \(\pi = \frac{1}{n} \mathbf{1}\). La dinamica Q converge sempre dal momento che vengno aggiunti self-loop ad ogni nodo e quindi il grafo risulta sempre aperiodico. In particolare, poiché:
   \[\pi' P = \pi' \iff \pi'Q = \pi'\]
   a parità di condizioni iniziali \(x(0)\) la dinamica \(Q\) converge allo stesso consenso della dinamica \(P\) (nel caso in cui \(\frac{n}{2}\) sia pari)

   \questionpart %---------------------B------------------------------
   \[P = \frac{1}{3} W \implies Q = \frac{1}{6}W + \frac{1}{2}I\]
   \[Q = \begin{bmatrix}
       \frac{1}{2} & \frac{1}{6} & \dots & \frac{1}{6} & \dots & \frac{1}{6} \\
       \frac{1}{6} & \frac{1}{2} & \frac{1}{6} & \dots & \frac{1}{6} & \dots \\
       \vdots 
   \end{bmatrix}\]
   \(Q\) è una matrice circolare la cui prima riga \(q\) è così composta:
   \[q_0 = \frac{1}{2},\quad q_1 = q_{\frac{n}{2}} = q_{n - 1} = \frac{1}{6}, \quad q_k = 0\,  \forall k\notin \{1, 2, \frac{n}{2}, n - 1\}.\]
   (gli indici sono shiftati di una posizione, da \(0\) a \(n - 1\)).
   Poiché \(Q\) è circolare allora lo spettro sarà dato da:
   \[\lambda_k =  \sum \limits_{j = 0}^{n - 1} q_j \omega_k^j,\quad k = 0, \dots , n - 1\]
   \[\omega_k = \exp\left[{\frac{2 \pi i }{n}k}\right].\]
   Pertanto:
   \[ \lambda_k = \frac{1}{2} \omega_k^0 + \frac{1}{6}\omega_k^1 + \frac{1}{6}\omega_k^{\frac{n}{2}} + \frac{1}{6} \omega_k^{n - 1} =
    \]
    \[ = \frac{1}{2} + \frac{1}{6}\exp \left[ \frac{2 \pi i }{n}k\right] + \frac{1}{6}\exp \left[ -\frac{2 \pi i }{n}k\right] + \frac{1}{6}\exp \left[ \pi i k\right] =\]
    \[ = \frac{1}{2} + \frac{1}{3} \cos \left(\frac{2 \pi}{n}k\right) + \frac{1}{6}\exp \left[ \pi i k\right]\]
    \[ \implies \lambda_1 = \frac{2}{3} + \frac{1}{3}\cos\left(\frac{2 \pi}{n}\right)\]
    \end{alphaparts}