\documentclass[11pt,largemargins]{homework}

\newcommand{\hwname}{Giulio Nenna}
\newcommand{\hwemail}{s292399@studenti.polito.it}
\newcommand{\hwtype}{Homework}
\newcommand{\hwnum}{2}
\newcommand{\hwclass}{}
\newcommand{\hwlecture}{}
\newcommand{\hwsection}{}

% This is just used to generate filler content. You don't need it in an actual
% homework!
\usepackage{lipsum}
\usepackage{amssymb}
\usepackage[utf8]{inputenc}
\usepackage[T1]{fontenc}
\usepackage{lmodern}
\usepackage{amsfonts}
\usepackage{hyperref}
\usepackage{bbm}
\usepackage{amsmath}
\usepackage{mcode}
\usepackage{epstopdf}
\usepackage{subcaption}
\usepackage{float}
\usepackage{bbm}
\usepackage{amsthm}
\usepackage{mathtools}
\usepackage{tikzit}
\usepackage{caption}
\usepackage{listings}
\usepackage{color} %red, green, blue, yellow, cyan, magenta, black, white
\definecolor{mygreen}{RGB}{28,172,0} % color values Red, Green, Blue
\definecolor{mylilas}{RGB}{170,55,241}
\input{Stilegrafo.tikzstyles}
\begin{document}


\lstset{language=Matlab,%
    %basicstyle=\color{red},
    breaklines=true,%
    morekeywords={matlab2tikz},
    keywordstyle=\color{blue},%
    morekeywords=[2]{1}, keywordstyle=[2]{\color{black}},
    identifierstyle=\color{black},%
    stringstyle=\color{mylilas},
    commentstyle=\color{mygreen},%
    showstringspaces=false,%without this there will be a symbol in the places where there is a space
    numbers=left,%
    numberstyle={\tiny \color{black}},% size of the numbers
    numbersep=9pt, % this defines how far the numbers are from the text
    emph=[1]{for,end,break},emphstyle=[1]\color{red}, %some words to emphasise
    %emph=[2]{word1,word2}, emphstyle=[2]{style},    
}
\maketitle

\begin{center}
  Realizzato in collaborazione con Alessandro Bonaduce (s289906@studenti.polito.it), Davide Grande (s292174@studenti.polito.it).
\end{center}


\section{}% --------------ESERCIZIO 1------------------------------
\begin{alphaparts}
    %-------------------------------------A---------------------------------------
   \questionpart
   Per la dinamica \(P\) (non lazy) distinguiamo 2 casi:
   \begin{description}
       \item[\(\bullet\)] Se \(\frac{n}{2}\) è dispari allora il grafo \textbf{non} è aperiodico, pertanto la convergenza della dinamica \(P\) non è garantita.
       \item[\(\bullet\)] Se \(\frac{n}{2}\) è pari allora siano 
       \[p_1 = \{0,1,2, \dots ,n - 1, 0\}\] 
       \[p_2 = \{0,1,2, \dots \frac{n}{2}, 0\}\]
       due cicli di lunghezza rispettivamente \(n\) e \(\frac{n}{2} + 1\).
       \[\frac{n}{2}  \quad \text{pari} \quad \implies\, n= 4k,\,k\in \mathbb{N}\]
       Abbiamo quindi trovato due cicli di lunghezza \(4k\) e \(2k + 1\) che sono numeri coprimi tra loro, pertanto il grafo è aperiodico e la convergenza è garantita. In particolare
       \[x(t) \xrightarrow[]{t \to \infty} (\pi 'x(0)) \mathbf{1}\]
   \end{description} 
   Poiché il grafo \(R_n\) è regolare, allora \(\pi = \frac{1}{n} \mathbf{1}\). La dinamica Q converge sempre dal momento che vengno aggiunti self-loop ad ogni nodo e quindi il grafo risulta sempre aperiodico. In particolare, poiché:
   \[\pi' P = \pi' \iff \pi'Q = \pi'\]
   a parità di condizioni iniziali \(x(0)\) la dinamica \(Q\) converge allo stesso consenso della dinamica \(P\) (nel caso in cui \(\frac{n}{2}\) sia pari)

   \questionpart %---------------------B------------------------------
   \[P = \frac{1}{3} W \implies Q = \frac{1}{6}W + \frac{1}{2}I\]
   \[Q = \begin{bmatrix}
       \frac{1}{2} & \frac{1}{6} & \dots & \frac{1}{6} & \dots & \frac{1}{6} \\
       \frac{1}{6} & \frac{1}{2} & \frac{1}{6} & \dots & \frac{1}{6} & \dots \\
       \vdots 
   \end{bmatrix}\]
   \(Q\) è una matrice circolare la cui prima riga \(q\) è così composta:
   \[q_0 = \frac{1}{2},\quad q_1 = q_{\frac{n}{2}} = q_{n - 1} = \frac{1}{6}, \quad q_k = 0\,  \forall k\notin \{1, 2, \frac{n}{2}, n - 1\}.\]
   (gli indici sono shiftati di una posizione, da \(0\) a \(n - 1\)).
   Poiché \(Q\) è circolare allora lo spettro sarà dato da:
   \[\lambda_k =  \sum \limits_{j = 0}^{n - 1} q_j \omega_k^j,\quad k = 0, \dots , n - 1\]
   \[\omega_k = \exp\left[{\frac{2 \pi i }{n}k}\right].\]
   Pertanto:
   \[ \lambda_k = \frac{1}{2} \omega_k^0 + \frac{1}{6}\omega_k^1 + \frac{1}{6}\omega_k^{\frac{n}{2}} + \frac{1}{6} \omega_k^{n - 1} =
    \]
    \[ = \frac{1}{2} + \frac{1}{6}\exp \left[ \frac{2 \pi i }{n}k\right] + \frac{1}{6}\exp \left[ -\frac{2 \pi i }{n}k\right] + \frac{1}{6}\exp \left[ \pi i k\right] =\]
    \[ = \frac{1}{2} + \frac{1}{3} \cos \left(\frac{2 \pi}{n}k\right) + \frac{1}{6}\exp \left[ \pi i k\right]\]
    \[ \implies \lambda_1 = \frac{2}{3} + \frac{1}{3}\cos\left(\frac{2 \pi}{n}\right)\]
    \end{alphaparts}

\section{}% --------------ESERCIZIO 2------------------------------

\begin{alphaparts}
    \questionpart %----------------A-------------------------------
   \ctikzfig{grafo_es2_a}
Posizionando il nodo stubborn sul nodo $n_0$ in figura (ovvero il nodo madre dell’albero) l’agente $A$ avrebbe la certezza che l’opinione finale di ogni agente sia esattamente $1$. Il nodo stubborn è infatti global reachable e ciò implica che l’opinione finale di tutti i nodi sarà una combinazione convessa del solo valore $1$. 
A questo punto il giocatore $B$ deve posizionare il suo nodo stubborn con valore $0$ in modo da rendere più vicina a $0$ possibile la media di opinioni finale.
Per farlo è sufficiente porre il suo nodo stubborn su $n_2$ ovvero sul figlio destro del nodo $1$. in questo modo l’opinione finale di tutti i nodi della parte destra del grafo sarà pari a $0$.
Questa è la scelta ottimale in quanto se posizionasse il nodo stubbron su $n_1$ (figlio sinistro di $n_0$) si avrebbero tutti i nodi della parte sinistra dell’albero con opinione finale nulla ma essi sono in numero minore rispetto ai nodi nella parte destra e ciò implicherebbe una media finale maggiore.
Se infine, il nodo stubborn venisse posizionato su uno qualsiasi degli altri nodi, l’opinione finale dei nodi compresi tra i due stubbron sarebbe una combinazione convessa dei valori di questi ultimi. Anche in questo caso, il giocatore $B$ non riuscirebbe ad ottenere il minimo valor medio di opinione possibile. 
   

    

    \questionpart %----------------B-------------------------------

Per le osservazioni al punto precedente il giocatore $B$ dovrebbe posizionare il nodo stubborn ancora una volta su $n_2$, in questo modo la scelta ottimale per il giocatore $A$ sarebbe necessariamente il nodo $n_1$, ma la media di opinioni finale sarà più vicina a zero per via della maggior presenza di nodi nella parte destra dell’albero.
 \ctikzfig{grafo_es2_b}

    
\end{alphaparts}


\section{}% --------------ESERCIZIO 3------------------------------

\begin{alphaparts}
    \questionpart %----------------A-------------------------------
    La comunità di agenti si presenta sotto forma del seguente grafo:
    \ctikzfig{grafo_es3}
    
    Dal momento che quest'ultima fortemente connesso, la dinamica delle opinioni raggiunge il consenso:
    \[x(t) \xrightarrow[]{t\to \infty}(\pi'x(0))\mathbbm{1}\]
    Dal momento che consideriamo \(x(0)\) come un vettore di variabili aleatorie indipendenti con le varianze date, allora esso avrà una matrice di varianza-covarianza così composta:
    \[VarCov(x(0)):= V= \begin{bmatrix}
        0.1 & 0 & 0 & 0 \\
        0 & 0.3 & 0 & 0\\
        0 & 0 & 0.2 & 0\\
        0 & 0 & 0 & 0.2
    \end{bmatrix}\]
    Pertanto, utilizzando la formula per la varianza di un prodotto scalare di una V.A. con un vettore costante, si ottiene:
    \[Var(\pi'x(0)) = \pi' V \pi.\]
    Si procede quindi con il calcolo di \(\pi\). Essendo il grafo fortemente connesso abbiamo:
    \[\pi = \frac{\omega}{\mathbbm{1}'\omega}\]
    pertanto:
    \[Var(\pi'x(0)) \sim 0.0519\]

    \questionpart %----------------B----------------------
    Sia \(V\) la matrice di varianza-covarianza di una comunità. La varianza del consenso \(\gamma\), se raggiunto, sarà data dalla formula:
    \[Var(\gamma) = \pi' V \pi\]
    dove \(\pi\) è la distribuzione stazionaria del grafo che connette la comunità. Costruire un grafo che minimizza la varianza del consenso (e quindi l'errore di misura nel caso di dinamiche di opinioni) significa trovare una \(\pi\) che risolve il seguente problema di minimo:
    \begin{align*}
        \text{minimize:} \quad & \pi'V\pi \\
        \text{subject to:} \quad &\mathbbm{1}'\pi = 1 \\
        & \pi \geq  0
    \end{align*}

    Risolvendo il problema senza considerare il constraint di non negatività si ottiene la funzione lagrangiana:
    \[\mathcal{L}(\pi, \lambda) = \pi'V\pi + \lambda(\mathbbm{1}'\pi- 1)\]
    e risolvendo il sistema:
    \begin{equation*}
        \begin{cases} \nabla_{\pi} \mathcal{L} = 2V\pi + \lambda \mathbbm{1} = 0 \\
        \nabla_{\lambda} \mathcal{L} = \mathbbm{1}'\pi- 1 = 0\end{cases} 
    \end{equation*}
    si trova la soluzione:
    \begin{equation}
        \pi^* = \frac{1}{\mathbbm{1}'V\mathbbm{1}} V^{- 1} \mathbbm{1}.
    \end{equation}
    Osserviamo che, poiché \(V\) in quanto matrice di varianza-covarianza, è semidefinita positiva, necessariamente \(\pi^*\) è un vettore non negativo, pertanto il constraint di non negatività è automaticamente soddisfatto.
\end{alphaparts}



\section{}% --------------ESERCIZIO 4------------------------------
\begin{alphaparts}
    \questionpart %----------------A-------------------------------
    
Definito $t_{k}$ come il tempo del $k$-esimo contagio, pertanto $\tilde{T}_{k} := t_{k+1} - t_{k}$ rappresenta il tempo di attesa per il prossimo contagio. È noto che $\tilde{T}_{k}$ condizionato ad $X(t)$ ha distribuzione esponenziale $\beta\,B(t_{k})$ e quindi
\[\mathbb{E} [ \tilde{T}_{k} \,\vert\, X(t_{k}) ] = \frac{1}{\beta\,B(t_{k})}.\]
Di conseguenza, partendo da un numero di $k_{0}$ contagiati, il tempo medio di assorbimento nella configurazione con tutti i nodi infetti, $\tau = \mathbb{E}[\mathbb{T}_{1}]$, si può scrivere come
\begin{align}
\tau &= \mathbb{E} [ \tilde{T}_{k_{0}} \,\vert\, X(t_{k_{0}}) ] +  \mathbb{E} [ \tilde{T}_{k_{0}+1} \,\vert\, X(t_{k_{0}+1}) ] + ... +  \mathbb{E} [ \tilde{T}_{n-1} \,\vert\, X(t_{n-1}) ] \nonumber\\
&= \sum_{k=k_{0}}^{n-1} \mathbb{E} [ \tilde{T}_{k} \,\vert\, X(t_{k}) ] \nonumber\\
&= \sum_{k=k_{0}}^{n-1} \frac{1}{\beta\,B(t_{k})}.\nonumber
\end{align}
Ad un dato istante di tempo $t_{k}$ è possibile suddividere l'insieme dei nodi $V$ in due sottoinsiemi:
\[ S_{k} : = \{ i : X_{i}(t_{k}) = 0 \} \]
\[ I_{k} : = \{ i : X_{i}(t_{k}) = 0 \} = V\setminus S_{k} \]
Si osserva ora che 
\begin{align}
B(t_{k})  &:= \sum_{i} \,\sum_{j} W_{ij}(1-X_{i}(t_{k}))X_{j}(t_{k}) \nonumber\\
&= \sum_{i} \sum_{j\in V\setminus S_{k} } W_{ij}(1-X_{i}(t_{k}))\nonumber\\
&= \sum_{i\in S_{k}} \sum_{j\in V\setminus S_{k} } W_{ij}.\nonumber
\end{align}
Si nota che \(\vert S_{k} \vert = n-k\), pertanto
\[\gamma(n-k) : = \min_{S\subset V : \vert S \vert = n - k} \sum_{i\in S_{k}} \sum_{j\in V\setminus S_{k} } W_{ij} \le \sum_{i\in S_{k}} \sum_{j\in V\setminus S_{k} } W_{ij} = B(t_{k}). \]
A questo punto ritornando ai conti fatti su $\tau$ si ha che
\begin{align}
\tau &=\sum_{k=k_{0}}^{n-1} \frac{1}{\beta\,B(t_{k})}\nonumber\\
& \le \frac{1}{\beta}\left(\frac{1}{\gamma(n-k_{0})}+\frac{1}{\gamma(n-k_{0}-1)}+ \dots +\frac{1}{\gamma(1)}\right)\nonumber\\
& \le \frac{1}{\beta} \sum_{k = 1}^{n-k_{0}} \frac{1}{\gamma(k)} \le \frac{1}{\beta} \sum_{k = 1}^{n-1} \frac{1}{\gamma(k)}.\nonumber
\end{align}
Ora si spezza la sommatoria su $k$ nel modo seguente
\[ \sum_{k = 1}^{n-1} \frac{1}{\gamma(k)} = \sum_{k = 1}^{\lfloor n/2\rfloor} \frac{1}{\gamma(k)} + \sum_{k = \lfloor n/2\rfloor}^{n-1} \frac{1}{\gamma(k)}.\]
A questo punto, se si prova che 
\[\sum_{k = 1}^{\lfloor n/2\rfloor} \frac{1}{\gamma(k)} \ge \sum_{k = \lfloor n/2\rfloor}^{n-1} \frac{1}{\gamma(k)} \]
si ottiene quanto dev'essere dimostrato nell'esercizio.

Poiché il grafo è \textit{undirected}, $W$ è simmetrica e quindi
\[\sum_{i\in S_{k}} \sum_{j\in V\setminus S_{k} } W_{ij} =  \sum_{j\in V\setminus S_{k} } \sum_{i\in S_{k}} W_{ji} = \sum_{i\in V\setminus S_{k} } \sum_{j\in S_{k}} W_{ij}. \]
Ponendo ora \( S^{c} := V\setminus S \) e osservando che se \( \vert S \vert = k\) allora \(\vert S^{c} \vert = n-k\) si può vedere che
\begin{align}
\gamma(k) &= \min_{S\subset V : \vert S \vert = k} \sum_{i\in S}\sum_{j\in S^{c}} W_{ij} \nonumber\\
&= \min_{S\subset V : \vert S \vert = k} \sum_{i\in S^{c}}\sum_{j\in S} W_{ij} \nonumber\\
&= \min_{S^{c}\subset V : \vert S^{c} \vert = n - k} \sum_{i\in S^{c}}\sum_{j\in S} W_{ij} \nonumber\\
&= \gamma(n-k). \nonumber
\end{align}
Quindi, in particolare,
\[ \gamma(k) \le \gamma(n-k) \]
da cui segue che
\[\sum_{k = 1}^{\lfloor n/2\rfloor} \frac{1}{\gamma(k)} \ge \sum_{k = \lfloor n/2\rfloor}^{n-1} \frac{1}{\gamma(k)}, \]
ossia
\[\sum_{k = 1}^{n - 1} \frac{1}{\gamma(k)} \le 2\,\sum_{k = 1}^{\lfloor n/2\rfloor} \frac{1}{\gamma(k)}. \]

    \questionpart %----------------B-------------------------------
Il grafo in questione è \textit{indiretto} e \textit{connesso}, pertanto si può utilizzare il risultato del punto (a) per ottenere la stima sul tempo di assorbimento 
\begin{align*}
\tau \leq \frac{2}{\beta} \sum_{k=1}^{\lfloor n/2\rfloor} \frac{1}{\gamma(k)}
\end{align*}
utilizzando la disuguaglianza
\begin{align*}
\gamma(k) \geq \sqrt{2k}, \vspace{0.5cm} k \in \{1,...,\lfloor n/2\rfloor \}
\end{align*}
si ottiene che 
\begin{align*}
\frac{2}{\beta} \sum_{k=1}^{\lfloor n/2\rfloor} \frac{1}{\gamma(k)} \leq \frac{2}{\beta} \sum_{k=1}^{\lfloor n/2\rfloor} \frac{1}{\sqrt{2k}}
\end{align*}
ovvero
\begin{align*}
\tau \leq \frac{\sqrt{2}}{\beta} \sum_{k=1}^{\lfloor n/2\rfloor} \frac{1}{\sqrt{k}}.
\end{align*}
Poiché $n=m^2$
\begin{align*}
\tau \leq \frac{\sqrt{2}}{\beta} \sum_{k=1}^{\lfloor m^2/2\rfloor} \frac{1}{\sqrt{k}}.
\end{align*}

A questo punto, per le note formule sulle sommatorie finite, si può concludere che
\begin{align*}
\tau \leq \frac{\sqrt{2}}{\beta} \sum_{k=1}^{\left \lfloor m^2/2\right \rfloor} k^{-1/2}  \asymp \frac{\sqrt{2}}{\beta} \left\lfloor \frac{m^2}{2}\right \rfloor^{1/2} .
\end{align*}


   \questionpart %----------------C-------------------------------
   
Si osserva che per il grafo bilanciere \textit{indiretto} il valore 
\begin{align*}
\sum_{i\in S} \sum_{j\in V\setminus S } W_{ij}
\end{align*}
non è altro che il numero di archi uscenti che collegano una partizione $S \subset V$ dei nodi con la partizione $V\setminus S$.
Per ognuno dei due sottografi $K_n$ del grafo bilanciere, purchè non si comprenda il nodo comunicante tra i due sottografi, vale che, per ogni partizione tale per cui $|S| = k$, si hanno $k(n-k)$ archi uscenti.
Matematicamente questo implica che 
\begin{align*}
\gamma (k) = 
\begin{cases}
k(n-k) , & \text{se $k=0,1,...,n-1$,} \\
1, & \text{se $k=n$.}
\end{cases}
\end{align*}

Di conseguenza, applicando la stima ricavata al punto $(a)$ si può scrivere
\begin{align*}
\tau \le \frac{2}{\beta} \sum_{k = 1}^{n} \frac{1}{\gamma(k)} = \frac{2}{\beta} \bigg( \sum_{k = 1}^{n-1} \frac{1}{k(n-k)} + 1 \bigg)
\end{align*}

da cui 

\begin{align*}
\tau \le \frac{4}{\beta}  \frac{H_{n-1}}{n} + \frac{2}{\beta} \asymp   \frac{4}{\beta}  \frac{log(n)}{n}.\end{align*}










    
\end{alphaparts}



% Sometimes questions get separated from their bodies. Use a \newpage to force
% them to wrap to the next page.

  
% Use \renewcommand{\questiontype}{<text>} to change what word is displayed
% before numbered questions
%\renewcommand{\questiontype}{Task}
\end{document}
