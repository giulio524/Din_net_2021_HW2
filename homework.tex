\documentclass[11pt,largemargins]{homework}

\newcommand{\hwname}{Giulio Nenna}
\newcommand{\hwemail}{s292399@studenti.polito.it}
\newcommand{\hwtype}{Homework}
\newcommand{\hwnum}{1}
\newcommand{\hwclass}{}
\newcommand{\hwlecture}{}
\newcommand{\hwsection}{}

% This is just used to generate filler content. You don't need it in an actual
% homework!
\usepackage{lipsum}
\usepackage{amssymb}
\usepackage[utf8]{inputenc}
\usepackage[T1]{fontenc}
\usepackage{lmodern}
\usepackage{amsfonts}
\usepackage{hyperref}
\usepackage{bbm}
\usepackage{amsmath}
\usepackage{mcode}
\usepackage{epstopdf}
\usepackage{subcaption}
\usepackage{float}
\usepackage{bbm}
\usepackage{amsthm}
\usepackage{mathtools}
\usepackage{tikzit}
\usepackage{caption}
\usepackage{listings}
\usepackage{color} %red, green, blue, yellow, cyan, magenta, black, white
\definecolor{mygreen}{RGB}{28,172,0} % color values Red, Green, Blue
\definecolor{mylilas}{RGB}{170,55,241}
\input{Stilegrafo.tikzstyles}
\begin{document}


\lstset{language=Matlab,%
    %basicstyle=\color{red},
    breaklines=true,%
    morekeywords={matlab2tikz},
    keywordstyle=\color{blue},%
    morekeywords=[2]{1}, keywordstyle=[2]{\color{black}},
    identifierstyle=\color{black},%
    stringstyle=\color{mylilas},
    commentstyle=\color{mygreen},%
    showstringspaces=false,%without this there will be a symbol in the places where there is a space
    numbers=left,%
    numberstyle={\tiny \color{black}},% size of the numbers
    numbersep=9pt, % this defines how far the numbers are from the text
    emph=[1]{for,end,break},emphstyle=[1]\color{red}, %some words to emphasise
    %emph=[2]{word1,word2}, emphstyle=[2]{style},    
}
\maketitle

\begin{center}
  Realizzato in collaborazione con Alessandro Bonaduce (s289906@studenti.polito.it), Davide Grande (s292174@studenti.polito.it), Ciro Balsamo (s289363@studenti.polito.it) 
\end{center}


\section{}% --------------ESERCIZIO 1------------------------------
\begin{alphaparts}
    %-------------------------------------A---------------------------------------
   \questionpart
   Per la dinamica \(P\) (non lazy) distinguiamo 2 casi:
   \begin{description}
       \item[\(\bullet\)] Se \(\frac{n}{2}\) è dispari allora il grafo \textbf{non} è aperiodico, pertanto la convergenza della dinamica \(P\) non è garantita.
       \item[\(\bullet\)] Se \(\frac{n}{2}\) è pari allora siano 
       \[p_1 = \{0,1,2, \dots ,n - 1, 0\}\] 
       \[p_2 = \{0,1,2, \dots \frac{n}{2}, 0\}\]
       due cicli di lunghezza rispettivamente \(n\) e \(\frac{n}{2} + 1\).
       \[\frac{n}{2}  \quad \text{pari} \quad \implies\, n= 4k,\,k\in \mathbb{N}\]
       Abbiamo quindi trovato due cicli di lunghezza \(4k\) e \(2k + 1\) che sono numeri coprimi tra loro, pertanto il grafo è aperiodico e la convergenza è garantita. In particolare
       \[x(t) \xrightarrow[]{t \to \infty} (\pi 'x(0)) \mathbf{1}\]
   \end{description} 
   Poiché il grafo \(R_n\) è regolare, allora \(\pi = \frac{1}{n} \mathbf{1}\). La dinamica Q converge sempre dal momento che vengno aggiunti self-loop ad ogni nodo e quindi il grafo risulta sempre aperiodico. In particolare, poiché:
   \[\pi' P = \pi' \iff \pi'Q = \pi'\]
   a parità di condizioni iniziali \(x(0)\) la dinamica \(Q\) converge allo stesso consenso della dinamica \(P\) (nel caso in cui \(\frac{n}{2}\) sia pari)

   \questionpart %---------------------B------------------------------
   \[P = \frac{1}{3} W \implies Q = \frac{1}{6}W + \frac{1}{2}I\]
   \[Q = \begin{bmatrix}
       \frac{1}{2} & \frac{1}{6} & \dots & \frac{1}{6} & \dots & \frac{1}{6} \\
       \frac{1}{6} & \frac{1}{2} & \frac{1}{6} & \dots & \frac{1}{6} & \dots \\
       \vdots 
   \end{bmatrix}\]
   \(Q\) è una matrice circolare la cui prima riga \(q\) è così composta:
   \[q_0 = \frac{1}{2},\quad q_1 = q_{\frac{n}{2}} = q_{n - 1} = \frac{1}{6}, \quad q_k = 0\,  \forall k\notin \{1, 2, \frac{n}{2}, n - 1\}.\]
   (gli indici sono shiftati di una posizione, da \(0\) a \(n - 1\)).
   Poiché \(Q\) è circolare allora lo spettro sarà dato da:
   \[\lambda_k =  \sum \limits_{j = 0}^{n - 1} q_j \omega_k^j,\quad k = 0, \dots , n - 1\]
   \[\omega_k = \exp\left[{\frac{2 \pi i }{n}k}\right].\]
   Pertanto:
   \[ \lambda_k = \frac{1}{2} \omega_k^0 + \frac{1}{6}\omega_k^1 + \frac{1}{6}\omega_k^{\frac{n}{2}} + \frac{1}{6} \omega_k^{n - 1} =
    \]
    \[ = \frac{1}{2} + \frac{1}{6}\exp \left[ \frac{2 \pi i }{n}k\right] + \frac{1}{6}\exp \left[ -\frac{2 \pi i }{n}k\right] + \frac{1}{6}\exp \left[ \pi i k\right] =\]
    \[ = \frac{1}{2} + \frac{1}{3} \cos \left(\frac{2 \pi}{n}k\right) + \frac{1}{6}\exp \left[ \pi i k\right]\]
    \[ \implies \lambda_1 = \frac{2}{3} + \frac{1}{3}\cos\left(\frac{2 \pi}{n}\right)\]
    \end{alphaparts}


\section{}% --------------ESERCIZIO 3------------------------------

\begin{alphaparts}
    \questionpart %----------------A-------------------------------
    La comunità di agenti si presenta sotto forma del seguente grafo:
    \ctikzfig{grafo_es3}
    
    Dal momento che quest'ultima fortemente connesso, la dinamica delle opinioni raggiunge il consenso:
    \[x(t) \xrightarrow[]{t\to \infty}(\pi'x(0))\mathbbm{1}\]
    Dal momento che consideriamo \(x(0)\) come un vettore di variabili aleatorie indipendenti con le varianze date, allora esso avrà una matrice di varianza-covarianza così composta:
    \[VarCov(x(0)):= V= \begin{bmatrix}
        0.1 & 0 & 0 & 0 \\
        0 & 0.3 & 0 & 0\\
        0 & 0 & 0.2 & 0\\
        0 & 0 & 0 & 0.2
    \end{bmatrix}\]
    Pertanto, utilizzando la formula per la varianza di un prodotto scalare di una V.A. con un vettore costante, si ottiene:
    \[Var(\pi'x(0)) = \pi' V \pi.\]
    Si procede quindi con il calcolo di \(\pi\). Essendo il grafo fortemente connesso abbiamo:
    \[\pi = \frac{\omega}{\mathbbm{1}'\omega}\]
    pertanto:
    \[Var(\pi'x(0)) \sim 0.0519\]

    \questionpart %----------------B----------------------
    Sia \(V\) la matrice di varianza-covarianza di una comunità. La varianza del consenso \(\gamma\), se raggiunto, sarà data dalla formula:
    \[Var(\gamma) = \pi' V \pi\]
    dove \(\pi\) è la distribuzione stazionaria del grafo che connette la comunità. Costruire un grafo che minimizza la varianza del consenso (e quindi l'errore di misura nel caso di dinamiche di opinioni) significa trovare una \(\pi\) che risolve il seguente problema di minimo:
    \begin{align*}
        \text{minimize:} \quad & \pi'V\pi \\
        \text{subject to:} \quad &\mathbbm{1}'\pi = 1 \\
        & \pi \geq  0
    \end{align*}

    Risolvendo il problema senza considerare il constraint di non negatività si ottiene la funzione lagrangiana:
    \[\mathcal{L}(\pi, \lambda) = \pi'V\pi + \lambda(\mathbbm{1}'\pi- 1)\]
    e risolvendo il sistema:
    \begin{equation*}
        \begin{cases} \nabla_{\pi} \mathcal{L} = 2V\pi + \lambda \mathbbm{1} = 0 \\
        \nabla_{\lambda} \mathcal{L} = \mathbbm{1}'\pi- 1 = 0\end{cases} 
    \end{equation*}
    si trova la soluzione:
    \begin{equation}
        \pi^* = \frac{1}{\mathbbm{1}'V\mathbbm{1}} V^{- 1} \mathbbm{1}.
    \end{equation}
    Osserviamo che, poiché \(V\) in quanto matrice di varianza-covarianza, è semidefinita positiva, necessariamente \(\pi^*\) è un vettore non negativo, pertanto il constraint di non negatività è automaticamente soddisfatto.
\end{alphaparts}



% Sometimes questions get separated from their bodies. Use a \newpage to force
% them to wrap to the next page.

  
% Use \renewcommand{\questiontype}{<text>} to change what word is displayed
% before numbered questions
%\renewcommand{\questiontype}{Task}
\end{document}
