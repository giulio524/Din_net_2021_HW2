\documentclass[11pt,largemargins]{homework}

\newcommand{\hwname}{Giulio Nenna}
\newcommand{\hwemail}{s292399@studenti.polito.it}
\newcommand{\hwtype}{Homework}
\newcommand{\hwnum}{1}
\newcommand{\hwclass}{}
\newcommand{\hwlecture}{}
\newcommand{\hwsection}{}

% This is just used to generate filler content. You don't need it in an actual
% homework!
\usepackage{lipsum}
\usepackage{amssymb}
\usepackage[utf8]{inputenc}
\usepackage[T1]{fontenc}
\usepackage{lmodern}
\usepackage{amsfonts}
\usepackage{hyperref}
\usepackage{bbm}
\usepackage{amsmath}
\usepackage{mcode}
\usepackage{epstopdf}
\usepackage{subcaption}
\usepackage{float}
\usepackage{bbm}
\usepackage{amsthm}
\usepackage{mathtools}
\usepackage{tikzit}
\usepackage{caption}
\input{Stilegrafo.tikzstyles}
\begin{document}
\maketitle

\begin{center}
  Realizzato in collaborazione con Alessandro Bonaduce (s289906@studenti.polito.it), Davide Grande (s292174@studenti.polito.it), Ciro Balsamo (s289363@studenti.polito.it) 
\end{center}


\section{}% --------------ESERCIZIO 1------------------------------
\begin{alphaparts}
    %----------------------------------------------------------------------------
   \questionpart
    
    \end{alphaparts}



% Sometimes questions get separated from their bodies. Use a \newpage to force
% them to wrap to the next page.

  
% Use \renewcommand{\questiontype}{<text>} to change what word is displayed
% before numbered questions
%\renewcommand{\questiontype}{Task}
\end{document}
